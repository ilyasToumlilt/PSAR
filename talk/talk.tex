%%%%%%%%%%%%%%%%%%%%%%%%%%%%%%%%%%%%%%%%%
% Beamer Presentation
% LaTeX Template
% Version 1.0 (10/11/12)
%
% This template has been downloaded from:
% http://www.LaTeXTemplates.com
%
% License:
% CC BY-NC-SA 3.0 (http://creativecommons.org/licenses/by-nc-sa/3.0/)
%
%%%%%%%%%%%%%%%%%%%%%%%%%%%%%%%%%%%%%%%%%

%----------------------------------------------------------------------------------------
%	PACKAGES AND THEMES
%----------------------------------------------------------------------------------------

\documentclass{beamer}
\usepackage[french]{babel}
\usepackage[utf8x]{inputenc}

\mode<presentation> {

% The Beamer class comes with a number of default slide themes
% which change the colors and layouts of slides. Below this is a list
% of all the themes, uncomment each in turn to see what they look like.

%\usetheme{default}
%\usetheme{AnnArbor}
%\usetheme{Antibes}
%\usetheme{Bergen}
%\usetheme{Berkeley}
%\usetheme{Berlin}
%\usetheme{Boadilla}
\usetheme{CambridgeUS}
%\usetheme{Copenhagen}
%\usetheme{Darmstadt}
%\usetheme{Dresden}
%\usetheme{Frankfurt}
%\usetheme{Goettingen}
%\usetheme{Hannover}
%\usetheme{Ilmenau}
%\usetheme{JuanLesPins}
%\usetheme{Luebeck}
%\usetheme{Madrid}
%\usetheme{Malmoe}
%\usetheme{Marburg}
%\usetheme{Montpellier}
%\usetheme{PaloAlto}
%\usetheme{Pittsburgh}
%\usetheme{Rochester}
%\usetheme{Singapore}
%\usetheme{Szeged}
%\usetheme{Warsaw}

% As well as themes, the Beamer class has a number of color themes
% for any slide theme. Uncomment each of these in turn to see how it
% changes the colors of your current slide theme.

%\usecolortheme{albatross}
\usecolortheme{beaver}
%\usecolortheme{beetle}
%\usecolortheme{crane}
%\usecolortheme{dolphin}
%\usecolortheme{dove}
%\usecolortheme{fly}
%\usecolortheme{lily}
%\usecolortheme{orchid}
%\usecolortheme{rose}
%\usecolortheme{seagull}
%\usecolortheme{seahorse}
%\usecolortheme{whale}
%\usecolortheme{wolverine}

%\setbeamertemplate{footline} % To remove the footer line in all slides uncomment this line
%\setbeamertemplate{footline}[page number] % To replace the footer line in all slides with a simple slide count uncomment this line

\setbeamertemplate{navigation symbols}{} % To remove the navigation symbols from the bottom of all slides uncomment this line

\setbeamertemplate{headline}{} % To remove headline uncomment this line
}

\usepackage{graphicx} % Allows including images
\usepackage{booktabs} % Allows the use of \toprule, \midrule and \bottomrule in tables

%----------------------------------------------------------------------------------------
%	TITLE PAGE
%----------------------------------------------------------------------------------------

\title[Soutenance PSAR]{Application embarquée sur carte pour l'automobile} % The short title appears at the bottom of every slide, the full title is only on the title page

\author{M. Bittan, R. Gouicem, I. Toumlilt} % Your name
\institute[UPMC] % Your institution as it will appear on the bottom of every slide, may be shorthand to save space
{
UPMC \\ % Your institution for the title page
\medskip
\textit{Encadrants: A. Blin, G. Muller, J. Sopena} % Your email address
}
\date{11 mai 2015} % Date, can be changed to a custom date

\titlegraphic{
  \includegraphics[height=0.6cm]{include/logo_upmc.png}%  % Include a department/university logo - this will require the graphicx package
  \hspace{1cm}%
  \includegraphics[height=0.9cm]{include/logo_lip6.png}% % Include a department/university logo - this will require the graphicx package
  \hspace{1cm}%
  \includegraphics[height=0.9cm]{include/logo_renault.png}}


\begin{document}

\begin{frame}
\titlepage % Print the title page as the first slide
\end{frame}

\begin{frame}
\frametitle{Contexte}
\begin{itemize}
  \item Automobile = système distribué
    \vspace{1em}
  \item Exécution de tâches à criticités mixtes
    \begin{itemize}
      \item Tâches \textit{best-effort}: GPS, infotainment
      \item Tâches \textit{temps-réel}: ABS, airbag, régulateur de distance
    \end{itemize}
    \vspace{1em}
  \item Assurer la sécurité et la fiabilité du système
\end{itemize}
\end{frame}

%---------------------------------------------------------

\begin{frame}
\frametitle{Contexte (2)}
\begin{center}
\includegraphics[scale=0.3]{include/archi.pdf}
\end{center}
\end{frame}

\begin{frame}
\frametitle{Objectifs}
\begin{itemize}
  \item Choix de l'application
  \item Parallèlisation de l'application
  \item Profilage
\end{itemize}
\end{frame}

\begin{frame}
\frametitle{Choix de l'application}
\begin{itemize}
  \item Beaucoup d'accès mémoire
  \item Trouvable dans une voiture
\end{itemize}
\end{frame}

\begin{frame}
\frametitle{Critères de selection}
\begin{itemize}
  \item Langage
  \item Nombre de dépendences
  \item Consommation mémoire
  \item Parallélisable
\end{itemize}
\end{frame}


\begin{frame}
\frametitle{Open Street Routing Machine}
Inconvénients :
\begin{itemize}
\item Nombre de dépendences élevé
\item Consommation mémoire trop élevé
\end{itemize}
\end{frame}


\begin{frame}
\frametitle{Routino}
\begin{itemize}
\item Aucune dépendence
\item Peu gourmand en mémoire
\item ... mais pas parallèle
\end{itemize}
\end{frame}



\begin{frame}
\frametitle{Algorithme de Routino}
\begin{itemize}
\end{itemize}
\end{frame}


<<<<<<< HEAD
%------------------------------------------------
\section{Second Section}
%------------------------------------------------

\begin{frame}
\frametitle{Table}
\begin{table}
\begin{tabular}{l l l}
\toprule
\textbf{Treatments} & \textbf{Response 1} & \textbf{Response 2}\\
\midrule
Treatment 1 & 0.0003262 & 0.562 \\
Treatment 2 & 0.0015681 & 0.910 \\
Treatment 3 & 0.0009271 & 0.296 \\
\bottomrule
\end{tabular}
\caption{Table caption}
\end{table}
\end{frame}

%------------------------------------------------

\begin{frame}
\frametitle{Theorem}
\begin{theorem}[Mass--energy equivalence]
$E = mc^2$
\end{theorem}
\end{frame}

%------------------------------------------------

\begin{frame}[fragile] % Need to use the fragile option when verbatim is used in the slide
\frametitle{Verbatim}
\begin{example}[Theorem Slide Code]
soguvhsqn
\end{example}
\end{frame}

%------------------------------------------------

\begin{frame}
\frametitle{Figure}
Uncomment the code on this slide to include your own image from the same directory as the template .TeX file.
%\begin{figure}
%\includegraphics[width=0.8\linewidth]{test}
%\end{figure}
\end{frame}
=======
% --------------------------------------------------

\begin{frame}
  \frametitle{Parallélisation de Routino}
  \begin{itemize}
  \item Séquentiel par portions
    \vspace{1em}
  \item Utilisation des données des précédentes portions
  \end{itemize}
\end{frame}

% --------------------------------------------------

\begin{frame}
  \frametitle{Parallélisation de Routino}

  \begin{itemize}
  \item Tour de France : En séquentiel
  \end{itemize}

  \begin{center}
    \includegraphics[scale=0.33]{include/tourfrance_mono.png}
  \end{center}

\end{frame}

% --------------------------------------------------

\begin{frame}
  \frametitle{Parallélisation de Routino}
  
  \begin{itemize}
  \item Tour de France : En Multithread
  \end{itemize}

  \begin{center}
    \includegraphics[scale=0.33]{include/tourfrance_multi.png}
  \end{center}
  
\end{frame}

% --------------------------------------------------

\begin{frame}
  \frametitle{Parallélisation de Routino}

  \begin{itemize}
  \item Rendre les portions indépendantes/
    \vspace{1em}
  \item Un minimum de synchronisation
  \end{itemize}

\end{frame}

% --------------------------------------------------

\begin{frame}
  \frametitle{Parallélisation de Routino}

  \begin{center}
    \only<1>{\includegraphics[scale=0.34]{include/multi1.pdf}}
    \only<2>{\includegraphics[scale=0.34]{include/multi2.pdf}}
    \only<3>{\includegraphics[scale=0.34]{include/multi3.pdf}}
  \end{center}

\end{frame}

% --------------------------------------------------

\begin{frame}
  \frametitle{Evaluation du Multithread}

  \begin{itemize}
  \item Usage CPU
  \end{itemize}

  \begin{center}
    \includegraphics[scale=0.24]{include/cpu_usage.png}
  \end{center}

\end{frame}

% --------------------------------------------------
>>>>>>> d0e4d77a3b88ddd517a0b32941ba2cec9ae915e8


%------------------------------------------------

\begin{frame}
  \frametitle{\'Etude d'impact sur une tâche temps-réel}
  \framesubtitle{Méthodologie}
  \begin{enumerate}
  \item<1-> Métriques
    \begin{itemize}
    \item Temps d'exécution
    \item Bande passante utilisée sur le système
    \end{itemize}
  \item<2-> Tâches temps-réel
    \begin{itemize}
    \item Benchmark MiBench
    \item Sous-ensemble d'applications représentatives d'un système automobile
    \end{itemize}
  \item<3-> Jeu de tests
    \begin{itemize}
    \item Tâche temps-réel en isolation
    \item Tâche temps-réel vs Routino (3 instances)
    \item Tâche temps-réel vs Routino MT (3 threads)
    \end{itemize}
  \item<4> Protocole
    \begin{itemize}
    \item[$T_0$] Lancement du/des Routino(s) si nécessaire (durée $\approx$
      45-80 s)
    \item[$T_0+\Delta$] Lancement de la tâche temps-réel (durée $<$ 100 ms)
    \end{itemize}
  \end{enumerate}
\end{frame}

%------------------------------------------------

\begin{frame}
  \frametitle{\'Etude d'impact sur une tâche temps-réel}
  \framesubtitle{Impact sur le temps d'exécution}
  \begin{center}
    \includegraphics[scale=0.18]{include/overhead.png}
  \end{center}
  \visible<2>{
    \begin{exampleblock}{Observations}
      \begin{itemize}
      \item Augmentation de 2 à 20\% du temps d'exécution
      \item \'Ecart de surcoût non constant
      \end{itemize}
    \end{exampleblock}
  }
\end{frame}

%------------------------------------------------

\begin{frame}
  \frametitle{\'Etude d'impact sur une tâche temps-réel}
  \framesubtitle{Impact sur la bande passante}
  \begin{center}
    \includegraphics[scale=0.175]{include/bandwidth.png}
  \end{center}
  \visible<2>{
    \begin{exampleblock}{Observations}
      La bande passante utilisée est similaire pour toutes les tâches du 
      benchmark. Les différences sont notamment dûes aux durées d'exécution
      de chacune.
    \end{exampleblock}
  }
\end{frame}

%------------------------------------------------

\begin{frame}
  \frametitle{\'Etude d'impact sur une tâche temps-réel}
  \framesubtitle{Corrélation bande passante/overhead}
  \begin{center}
    \includegraphics[scale=0.14]{include/correl.png}
  \end{center}
  \visible<2>{
    \begin{exampleblock}{Observation}
      Progression de l'overhead quasi-linéaire pour chaque type
      d'exécution (avec 3 instances de Routino ou avec RoutinoMT sur 3 threads)
    \end{exampleblock}
  }
\end{frame}

%------------------------------------------------

%---------------------------------------------------------------------------


%---------------------------------------------------------

\begin{frame}
\frametitle{Conclusion}
\begin{enumerate}
  \visible<1->{
  \item Réalisations
    \begin{itemize}
    \item Sélection d'une application
    \item Portage et parallélisation de l'application
    \item \'Etude de la parallélisation
    \item \'Etude de l'impact sur une tâche temps-réel MiBench
    \end{itemize}
  }
  \visible<2->{
  \item Travail à court terme
    \begin{itemize}
      \item Vérifier si l'overhead diminue avec l'algorithme d'ordonnancement
        d'Antoine Blin
      \item Soumettre un patch pour Routino
    \end{itemize}
  }
  \visible<3->{
    \item Travail à long terme
      \begin{itemize}
        \item \'Etendre la parallélisation aux trajets sans étapes
      \end{itemize}
  }
\end{enumerate}
\end{frame}

\end{document}
