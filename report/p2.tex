\section{Les applications}

L'un des principaux but du projet était de porter une application respectant 
certaines contraintes sur la carte présentée précedemment. En effet, la 
première contrainte était que le programme devait être capable de générer un 
trafic mémoire important sur le bus pendant une période de temps assez courte 
(environ une seconde). L'autre contrainte qu'il fallait respecter était que 
l'application devait pouvoir se trouver dans n'importe quelle voiture moderne. 
Ce sont ces deux raisons qui nous ont amené à choisir une application de calcul 
de plus court chemin basée sur des cartes OpenStreetMap. Il existe deux raisons 
qui font que nous avons choisi ces cartes. D'une part elles sont libres 
d'utilisation et d'autre part ce sont les cartes gratuites les plus 
exhaustives. 

\subsection{Selection de l'application}

L'application que nous avons décidé de porter sur la carte a été choisie selon 
un certain nombre de critères. \\

Tout d'abord, il fallait que celle-ci ne soit pas écrite en Java (comme c'est 
le cas pour de nombreuses applications utilisant les cartes OpenStreetMap) car 
le profiling mémoire aurait été dans ce cas plus difficile à effectuer à cause 
de la JVM. En effet, cette dernière effectue des optimisation au niveau de la 
mémoire qui nous aurait empêché de quantifier les besoins réels de 
l'application en terme de bande passante sur le bus. De plus, la quantité de 
mémoire disponible sur la carte étant relativement restreinte (1 Gio), nous ne 
pouvions pas nous permettre d'utiliser un language à ramasse-miettes comme 
Java sachant que les cartes OpenStreetMap sont elles aussi très gourmande en 
mémoire. \\

Ensuite, un autre critère de séléction pour l'application fût le nombre de 
dépendances de la partie du programme s'occupant du calcul du plus court 
chemin. En effet, bien que ce ne soit pas gênant d'un point de vue des 
performances ou des accès mémoire, les dépendances rendent le portage de 
l'application plus complexe. Il était donc utile, dans la limite du possible, 
de trouver un programme avec un petit nombre de dépendences. \\

Enfin, le dernier critère de selection fut le fait que l'application soit 
multithreadé (ou qu'elle puisse l'être assez simplement). En effet, afin 
d'étudier l'impact de l'application sur la tâche temps réel, il faut pouvoir 
l'exécuter sur les trois coeurs du processeur qui ne sont pas occupés par la
tache temps réel. Il aurait été possible de lancer trois instances d'un 
programme contenant un seul thread, mais cela ne nous a pas semblé realiste 
vis-à-vis du scénario que nous considérons. En effet, dans une voiture moderne, 
l'application GPS n'est pas exécuté trois fois, mais elle contient plusieurs 
fils d'exécutions qui effectuent une partie du calcul. \\

En tenant compte des contraintes énoncées précédemment, nous avons trouvé deux
applications : Open Source Routing Machine (OSRM) et Routino. L'application que
nous avons porté sur la carte est Routino, pour des raisons qui seront énoncés
plus tard dans le rapport.

\subsection{Open Source Routing Machine}

Un des deux programmes que nous avons examiné de près est Open Source Routing 
Machine (http://project-osrm.org/). Cette application est une implémentation 
très performante d'un moteur de calcul de plus court de chemin en C++. 
Concrètement, c'est un serveur multithreadé capable de gérer plusieurs calculs 
de chemins simultanément. Cependant, plusieurs raisons ont fait que nous 
n'avons pas choisi de porter cette application sur la carte. \\

Tout d'abord, le nombre de dépendences du programme est assez élevé (environ 
6 bibliothèques), ce qui rentre en contradiction avec ce qui à été dit 
précédemment et de ce fait rendait le portage de l'application inutilement
compliqué. \\

Ensuite, le programme qui effectue le calcul du plus court chemin est très 
gourmand en mémoire. En effet, les cartes OpenStreetMap sont assez consequentes 
($\sim$ 4.6 Gio pour la france compressé en bz2, et 22 Gio pour l'europe). C'est
 la raison pour laquelle toutes les applications possedent un outil permettant 
de convertir ces cartes en un format plus léger et directement compréhensible 
par le programme qui va effectuer le calcul du plus court chemin. Cependant, 
dans le cas d'OSRM, cet outil ne produit pas des cartes assez légeres, ce qui 
fait que le programme qui effectue le routage consomme environ 3 Gio de mémoire 
vive pour la carte de France. On ne peut donc pas utiliser ce programme avec la 
carte de France sur la carte embarquée, car celle-ci ne contient qu'un Gio de 
mémoire vive, et on veut que le programme n'entraine pas de swap. L'utilisation 
d'une carte plus petite (comme celle de l'Ile-de-France, par exemple) n'aurait 
pas aidé à regler le problème. En effet, comme le programme est très performant,
 les temps d'exécutions sur la carte aurait été trop courts par rapport à ceux 
de la tache temps réel. \\

Enfin, la façon dont l'application était parallélisée ne nous convenait pas. En 
effet, OSRM permet de lancer plusieurs calculs de plus court chemin en même 
temps mais l'algorithme effectuant ce calcul n'est lui-même pas parallele. 
La parallélisation de cet algorithme nous a semblé peu pertinente au vu de la 
complexité de celui-ci et des inconvenients que nous avons vu précédemment.

\subsection{Routino}

L'autre application que nous avons étudié de manière approfondie est Routino 
(www.routino.org). C'est l'application que nous avons décidé de porter sur la 
carte afin de pouvoir réaliser le profiling que nous verrons plus tard.

\subsubsection{Présentation de l'application}

Routino est une application écrite en C qui permet de calculer le chemin le 
plus court (ou le chemin le plus rapide) entre deux points. Cette application 
n'étant pas multithreadé, nous avons du effectuer le travail de parallélisation,
comme nous le verrons plus tard dans le rapport.

\paragraph{Raisons du choix de l'application}
\subparagraph{}

Il existe plusieurs facteurs qui nous ont poussé à choisir cette application 
plutôt qu'une autre. \\

D'une part le nombre de dépendences du programme effectuant le routage 
était nul, ce qui rendait le portage de l'application trivial. En effet, il 
suffit de changer le compilateur normal par le cross-compilateur de notre choix 
dans le makefile pour produire un exécutable compatible avec la carte. \\

D'autre part l'outil permettant de convertir les données OpenStreetMap en 
fichiers compréhensible par l'application effectuant le routage est assez 
efficace. En effet, pour la carte de France, les fichiers produits font environ 
860 Mio et peuvent donc rentrer sans problème dans la mémoire de la carte. Cela 
nous permet de calculer des plus courts chemins sur des distances plus 
conséquentes et donc d'augmenter le temps d'exécution de l'application. \\

\paragraph{Fonctionnement général de l'application}
\subparagraph{}

Avant de pouvoir effectuer des calculs de plus court chemins, il faut dans un 
premier temps convertir les cartes OpenStreetMap récupérées sur internet en des 
fichiers compréhensible par l'application de routage. C'est le rôle de 
l'exécutable nommé \texttt{planetsplitter}, qui produit les quatre fichiers 
suivant :

\begin{itemize}
\item \texttt{nodes.mem} : C'est le fichier qui contient les noeuds du graphe.
\item \texttt{ways.mem} : C'est le fichier qui contient des informations 
relatives aux arcs du graphe (type de transport autorisé sur l'arc, ...).
\item \texttt{segments.mem} : C'est le fichier qui contient les arcs du graphe.
\item \texttt{relations.mem} : Fichier qui contient des informations sur les 
relations entre les arcs du graphe. \\
\end{itemize}

Une fois ces fichiers obtenus, on peut appeler le programme \texttt{router} qui 
est capable d'effectuer des calculs de plus court chemin entre deux points. 
Pour réaliser cette opération, le programme commence par mapper les fichiers 
vus ci-dessus en mémoire. Si on a découpé le trajet en plusieurs morceaux dans
l'appel à \texttt{routeur} (par exemple si on indique qu'on veut le chemin le 
plus rapide de Paris à Nice en passant par Pau), le programme va calculer 
séquentiellement les chemins les plus rapides bout par bout. Dans notre exemple 
cela veut dire que l'application va d'abord calculer le chemin le plus rapide de 
Paris à Pau, puis de Pau à Nice. Cela peut sembler évident, mais il est 
important d'avoir ceci à l'esprit lorsque l'on abordera la parallélisation de 
l'application. L'algorithme utilisé par l'application est l'algorithme A*. \\

Le fait que le programme mappe les fichiers produit par \texttt{planetsplitter} 
peut entrainer des résultats peu fiables lors de la réalisation des benchmarks
 (seulement lors de la premiere execution de Routino). En effet, lorsqu'un 
programme mappe des fichiers en mémoire, le système d'exploitation (ici Linux) 
peut ne pas remplir la mémoire avec ces fichiers, et attendre que l'application 
accéde aux zones de mémoires concernées avant d'effectuer les entrées/sorties 
visant à rapatrier les fichiers en mémoire. De ce fait, lors de la première 
exécution du routeur, les temps d'exécution sont plus grands car les fichiers 
ne sont pas encore présent en mémoire. Cependant, lors des exécutions suivantes,
les temps sont plus faibles (et stables), car les données sont présentes dans le
 cache du système. Il faut donc s'assurer de bien exécuter Routino une première 
fois avant de lancer les scripts de benchmarks (voir l'annexe B pour plus 
d'informations). 
 

\paragraph{Utilisation de l'application}
\subparagraph{}

L'utilisation des programmes \texttt{planetsplitter}, \texttt{router} et 
\texttt{router\_multi} est expliquée en détail dans l'annexe A.

\subsubsection{Parallélisation}

\subsection{Les tâches temps réel}

Les taches temps réel que nous avons utilisés lors de notre projet sont issues 
de la suite logicielle \texttt{MiBench} (wwweb.eecs.umich.edu/mibench/). Les 
programmes dont nous nous sommes servis sont les suivants : 

\begin{itemize}
\item \texttt{qsort} : Cette tâche effectue un quicksort sur le fichier qui lui 
est donné en entrée. Dans notre cas, le fichier contient un enemble de nombres.
\item \texttt{susan} : Cette tâche est un programme de reconnaissance d'images 
capable de reconnaitre les coins et les bordures. Nous utilisons le programme 
avec le grand jeu de donnée fourni avec la suite MiBench (une image complexe 
en noir et blanc).
\item \texttt{patricia} : Les Patricia tries sont des arbres préfixes obtenus 
en fusionnant chaque noeuds n'ayant qu'un fils avec celui-ci. Cette structure 
est particulièrement adaptée pour la recherche d'elements. Ce programme effectue
 des ajouts et des recherches dans des Patricia tries. Le jeu de donnée que nous
 utilisons est un ensemble d'adresses IP ayant effectué des requêtes sur un 
serveur web assez fréquenté.
\item \texttt{aes} : Programme qui effectue un cryptage/décryptage de donnée 
avec AES. Le jeu de donnée utilisé en entrée est un texte ASCII trouvé en ligne.
\item \texttt{fft} : Programme qui effectue une transformée de Fourier rapide (et 
son inverse). 
\end{itemize}
