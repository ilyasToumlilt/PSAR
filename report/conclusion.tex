\section*{Conclusion}
Dans ce projet, nous avons pu mesurer l'impact des tâches \textit{best effort}
sur la tâche temps-réel. Ainsi, l'analyse temporelle et l'analyse de la bande
passante ont permis de mesurer le surcoût généré par cette exécution, mettant
en lumière un problème critique pour des tâches supposées s'exécuter en temps
réel. La tâche \textit{best effort} a donc un impact sur le bon fonctionnement
du système, augmentant les risques, dans le cas de l'automobile, pour les
passagers du véhicule en retardant les tâches à criticité élevée.
\paragraph{}
Cette conclusion ne peut qu'appuyer l'approche de Blin et al.
\cite{blin_protecting_2015} qui repose sur une analyse des besoins mémoire de
l'application temps-réel en amont, afin de suspendre les applications 
\textit{best effort} uniquement lorsque le risque de retarder les applications
critiques est excessif, permettant ainsi, dans la grande partie des cas, de
faire cohabiter les tâches de tous niveaux de criticité.
