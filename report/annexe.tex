\begin{appendices}
  \section{Utilisation de Routino}
  \label{ann:routino}

  Cette annexe vise à expliquer l'utilisation des différents programmes présents
  dans Routino.

  \subsection{planetsplitter}
  
  Le programme planetsplitter permet de convertir les cartes OpenStreetMap en 
  fichier légers directement compréhensible par le programme qui effectue le 
  routage. Le programme fonctionne de la façon suivante : \\
  
  \begin{lstlisting}
    ./planetsplitter [OPTIONS] [carte]  
  \end{lstlisting}
  \vspace{1em}

  Voici une petite liste des options importantes :
  
  \begin{itemize}
    \renewcommand{\labelitemi}{$\bullet$}
  \item \texttt{--dir=<directory>} : Dossier ou les fichiers de sortie seront 
    sauvegardés.
  \item \texttt{--tagging=<filename>} : Permet d'indiquer le chemin menant au 
    fichier \texttt{profiles.xml}.\\
  \end{itemize}

  La liste complete des options est disponible sur le site de Routino 
  \cite{bishop_routino_????}.
  
  \subsection{router / router\_multi}
  
  L'application permettant d'effectuer des calculs de plus court chemin est 
  \texttt{router}. La version parallèle de l'application est 
  \texttt{router\_multi}. Le programme s'utilise de la façon suivante : \\

  \begin{lstlisting}
    ./router [OPTIONS]
    ./router_multi [OPTIONS]
  \end{lstlisting}
  \vspace{1em}

  Voici quelques options importantes utilisable avec les deux programmes : \\
  
  \begin{itemize}
    \renewcommand{\labelitemi}{$\bullet$}
  \item \texttt{--dir=<directory>} : Dossier ou se situe les fichiers *.mem
    produits par \texttt{planetsplitter}.
  \item \texttt{--profiles=<file>} : Permet d'indiquer ou se situer le 
    fichier \texttt{profiles.xml}.
  \item \texttt{--translations=<file>} : Chemin menant au fichier 
    \texttt{translations.xml}.
  \item \texttt{--transport=<transport>} : Permet de spécifier le type de 
    transport à utiliser. Cette option peut prendre une des valeurs suivantes 
    : foot, horse, wheelchair, bicycle, moped, motorcycle, motorcar.
  \item \texttt{--lonXX, --latXX} : Permet de spécifier les coordonnées des 
    points composant le trajet. Le nombre minimum de points est 2. \\
  \end{itemize}

  Un liste plus complète des options de l'application \texttt{router} est
  disponible sur le site de Routino \cite{bishop_routino_????}. En plus des 
  options présentes dans \texttt{router}, \texttt{router\_multi} dispose d'une 
  option \texttt{--threads=N} permettant d'indiquer le nombre de thread que le 
  programme peut utiliser.
  
  \section{Scripts}
  \label{ann:scripts}
  
  Cette annexe vise à présenter les différents scripts que nous avons écrit 
  afin de produire les graphes vus précédemments.

  \subsection{cpusage}
  \label{ann:cpusage}
  
  Ce script permet de produire deux fichiers (cpusage.out et threadusage.out)
  qui servent de base a la production de deux graphes : l'utilisation des 
  differents coeurs du processeur au cours de l'exécution du routeur parallèle, 
  et l'utilisation du processeur par les threads au cours du temps. Il existe un
  certain nombre de variables globales au début du script permettant de definir 
  les chemins des fichiers nécessaire au bon fonctionnemment des scripts. Le 
  programme fonctionne de la façon suivante : on lance le routeur multithreadé 
  avec un itinéraire assez long. On appelle top en mode batch afin d'avoir les 
  pourcentages d'utilisation des coeurs du processeur et ceux des threads. 
  Une fois l'exécution du routeur terminé, on appelle le script 
  \texttt{parse\_cpusage.pl} qui produit le fichier \texttt{cpusage.out} qui 
  sert de base au graphe de la figure \ref{fig:cpusage}. On appelle ensuite 
  un second script qui produit le fichier \texttt{threadusage.out} qui sert de 
  base à la figure \ref{fig:threadusage}. 
  
  \subsection{speedup}
  \label{ann:speedup}

  Ce script permet de produire le graphe de la figure \ref{fig:speedup}. 
  Ce programme permet de calculer le gain de la parallèlisation en fonction
  du nombre de points dans l'itinéraire. Il faut donc fournir à ce script un 
  fichier contenant la liste des points du parcours. Ce fichier doit avoir le 
  format suivant :
  \begin{lstlisting}
    LATITUDE LONGITUDE
  \end{lstlisting}
  \vspace{1em}

  Le script fonctionne de la manière suivante : on lit le fichier contenant les 
  coordonnées des points. A chaque fois qu'on lit un point, on exécute les deux 
  versions du routeur en mesurant leur temps d'exécutions avec le point rajouté 
  à l'itinéraire précédent. 

  Le script peut être appelé avec les options suivantes : \\
  \begin{itemize}
    \renewcommand{\labelitemi}{$\bullet$}
  \item \texttt{--help} : Affiche l'aide.
  \item \texttt{--xmldir=<dir>} : Dossier contenant les fichiers xml.
  \item \texttt{--memdir<dir>} : Dossier contenant les fichiers produits
    par \texttt{planetsplitter}.
  \item \texttt{--waypoints=<file>} : Fichier contenant les coordonnées des 
    points composant le parcours. Cette option est obligatoire si on ne 
    modifie pas les variables dans le script.
  \item \texttt{--router=<file>} : Chemin vers le routeur non parallèle.
  \item \texttt{--routermt=<file>} : Chemin menant au routeur parallèle.
  \item \texttt{--transport=<type>} : Type de transport.
  \item \texttt{--output=<file>} : Fichier de sortie du script.
  \item \texttt{--threads=<number>} : Nombre de threads utilisé par le routeur
    parallèle.
  \end{itemize}

  \subsection{bench.sh}

  Ce script réalise un ensemble de mesures : 
  \begin{itemize}
    \renewcommand{\labelitemi}{$\bullet$}
  \item Durée d'exécution (en isolation et en concurrence) des tâches temps réel
  \item Surcout du temps d'exéuction lorsqu'exécuté en concurrence.
  \item Trafic mémoire (lecture et écriture).
  \end{itemize}

  Des fichiers de données pouvant être traité par gnuplot sont produits. Des
  scripts de traçage pour gnuplot sont disponibles (extension .plot). 

  \subsection{bandwidth.sh}

  Ce script prend les sorties du script précédent (bench.sh) et produit un 
  fichier de donnée pour gnuplot reflétant la bande passante utilisée en 
  moyenne sur le système durant l'exécution des tâches temps réel. 

\end{appendices}
