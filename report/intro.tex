\section*{Introduction}

Dans le monde de l'automobile, il n'est pas rare que des tâches à criticité 
mixte aient à s'exécuter sur l'un des processeurs présent dans le système. 
Cependant, il n'existe pour le moment aucun moyen d'assurer que les tâches les 
plus critiques (les tâches avec des contraintes temps réel par exemple) puissent
s'exécuter sans être génées par des tâches moins importantes. En effet, la 
quantité de données pouvant transiter sur le bus est limitée. De ce fait, si 
plusieurs tâches s'exécutent en même temps et font beaucoup d'accès sur le bus, 
il se peut que les performances de la tâche la plus critique diminuent. La 
solution utilisée par l'industrie actuellement est d'empêcher l'exécution de 
tâches best-effort quand des tâches temps réels s'exécutent. \\

Dans le cadre d'une thèse en collaboration avec Renault, le LIP6 et l'INRIA, 
Antoine Blin a proposé la solution suivante : les tâches best-effort peuvent 
s'exécuter en même temps que les tâches temps réel. Cependant, dès que les 
tâches peu critiques commencent à géner la tâche temps réel, on les suspend.\\

Les objectifs de notre projet étaient donc les suivants : dans un premier temps,
il fallait choisir une application à porter sur une carte embarquée qui nous a 
été fournie (une carte SABRE Lite). Dans un second temps, une fois ce travail 
terminé, nous devions effectuer le profilage de cette application. \\

Nous verrons tout d'abord dans ce rapport les spécifications de la carte, ainsi
que l'image système utilisée. Ensuite, nous aborderons les applications que nous
avons utilisé dans le cadre de ce projet. Enfin, nous présenterons le profilage de
l'application que nous avons choisi.
