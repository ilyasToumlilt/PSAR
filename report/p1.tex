\section{La carte SABRE Lite}

\subsection{Caractéristiques techniques}

\subsubsection{Architecture interne}
La carte SABRE Lite\cite{_i.mx_2014} embarque un processeur i.MX 6, basé sur le processeur ARM
Cortex A9 MPCore. Il s'agit d'un processeur quad-core cadencée à 1 GHz, 
disposant de caches L1 de 32 Ko ainsi que d'un cache L2 de 1 Mo partagé pour les
quatre coeurs. La carte dispose également de 1 Go de mémoire vive (DDR3). Le
contrôleur mémoire contient plusieurs compteurs permettant la mesure de 
l'activité mémoire qui seront utilisés dans la phase de profilage détaillée dans
la suite de ce rapport. Une puce de mémoire flash SPI de 2 Mo contenant un code
de boot est aussi présente.

\subsubsection{Stockage}
Le système ne dispose pas de stockage de masse interne intégré mais de ports
permettant l'ajout de périphériques de stockage :
\begin{itemize}
  \item un port SATA pour connecter un disque dur ou un SSD,
  \item deux ports USB 2.0,
  \item un port microUSB 2.0 supportant la norme OTG, afin de permettre au 
    périphérique de commander les échanges vers la carte,
  \item un port SD et un port microSD.
\end{itemize}

\subsubsection{Autres connectiques}
La carte dispose également d'un port Ethernet pour se connecter au réseau, ainsi
que d'un port UART, permettant une liaison série avec un ordinateur. Cette
liaison série sera notamment utilisée lors de la configuration de la carte. La
carte embarque également les ports suivants :
\begin{itemize}
  \item HDMI
  \item microphone et casque
  \item GPIO/I2C
  \item PCIe
  \item LVDS (Low-Voltage Differential Signaling)
  \item camera
\end{itemize}

\subsection{Image système}
Afin de rendre la carte utilisable, il est bien entendu nécessaire d'y installer
un système d'exploitation. Il est possible de booter la carte de deux manières:
\begin{itemize}
  \item via le port microUSB OTG,
  \item via la mémoire flash interne
\end{itemize}
Le mode de boot est contrôlé par un interrupteur sur la carte. Dans notre cas,
nous allons utiliser le code de boot préchargé dans la mémoire flash interne
pour ensuite charger l'image présente sur une carte microSD et démarrer sur
cette dernière.

\subsubsection{Construction de l'image}
Afin de construire l'image système pour la carte SABRE Lite, nous allons
utiliser la suite d'outils du projet Yocto. Il s'agit d'un projet géré par la
Linux Foundation et plusieurs grandes entreprises du monde de l'embarqué (Intel,
 Texas Instruments, Huawei, ...). Cette suite permet notamment la construction
d'images personnalisées pour systèmes embarqués. La construction se fait par
couches (\texttt{layers}), chacune représentant un ensemble logique de recettes
(\texttt{recipes}). Une recette est une suite d'instructions détaillant des 
étapes de construction (dépendance à d'autres recettes, origine des sources,
paquets à installer, ...).\cite{hallinan_create_2015}
\paragraph{} Après avoir concocté les couches et les recettes nécessaires à la
construction de l'image système (l'ajout d'un client \texttt{ssh} à une image
minimale dans notre cas), l'outil \texttt{bitbake}\cite{purdie_bitbake_????}
de la suite Yocto exécute toutes les recettes. Ce dernier construit entièrement
l'image pour le système ciblé, notamment en téléchargeant les sources de tous
les paquets et en les recompilant un à un. Des partitions (image kernel
\texttt{uImage}, rootfs), ainsi qu'une image pour carte SD contenant ces
dernières, sont alors générées. Une simple copie via un outil tel que
\texttt{dd} sur une carte microSD permet d'utiliser cette image système sur la
carte SABRE Lite.
